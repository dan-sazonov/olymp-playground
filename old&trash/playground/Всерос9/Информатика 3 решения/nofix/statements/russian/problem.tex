\begin{problem}{Без неподвижных точек}{стандартный ввод}{стандартный вывод}{1 секунда}{256 мегабайт}

Перестановкой $n$ элементов называется массив из различных натуральных $n$ чисел, каждое из которых от $1$ до $n$. Например, все перестановки $3$ элементов: $[1, 2, 3]$, $[1, 3, 2]$, $[2, 1, 3]$, $[2, 3, 1]$, $[3, 1, 2]$, $[3, 2, 1]$.

Элементы перестановки пронумерованы от одного до $n$, например для перестановки $a = [3, 1, 2]$ выполнено
$a[1] = 3$, $a[2] = 1$, $a[3] = 2$. Элемент с номером $i$ называется \textit{неподвижной точкой},
если $a[i] = i$. Так, в перестановке $[3, 1, 2]$ нет неподвижный точек, а в перестановке $[1, 3, 2]$ элемент
$a[1] = 1$ является неподвижной точкой.

Упорядочим все перестановки \textit{лексикографически} --- сначала по первому элементу,
потом по второму, и так далее. В начале условия все перестановки трех элементов приведены
в лексикографическом порядке. Оставим только те перестановки, которые не содержат неподвижных
точек. Для $n = 3$ останутся перестановки $[2, 3, 1]$ и $[3, 1, 2]$.

По заданным $n$ и $t$ требуется вывести первые $t$ в лексикографическом порядке
перестановок $n$ элементов без неподвижных точек. Перестановки следует выводить в лексикографическом порядке.

\InputFile
На ввод подаются два целых числа $n$ и $t$ ($2 \le n \le 1000$, $1 \le t \le 10^4$, $nt \le 10^5$). Гарантируется, что существует хотя бы $t$ перестановок $n$ элементов без неподвижных точек.

\OutputFile
Выведите $t$ строк, на $i$-й из них выведите $n$ чисел: $i$-ю в лексикографическом порядке перестановку $n$ элементов без неподвижных точек. 

\Scoring
Тесты в этой задаче разбиты на 7 групп. Баллы за группу начисляются только если
все тесты этой группы и всех необходимых групп пройдены.

Используется обозначение $F(n)$ для количества перестановок $n$ элементов без неподвижных точек.

\medskip

\begin{tabular}{|c|cc|c|l|}
\hline
№ & \multicolumn{2}{c|}{Ограничения}& Баллы & Условие начисления баллов\\
\hline
1 & $2 \le n \le 10$, $n$ четно& $t=1$ & 14 & Все тесты группы 1\\
\hline
2 & $2 \le n \le 1000$, $n$ четно & $t = 1$& 14& Все тесты групп 1 и 2\\
\hline
3 & $2 \le n \le 10$, $n$ нечетно& $t=1$ & 14 & Все тесты группы 3\\
\hline
4 & $2 \le n \le 1000$, $n$ нечетно & $t = 1$& 14& Все тесты групп 3 и 4\\
\hline
5 & $2 \le n \le 10$ & $t = F(n)$& 15 & Все тесты группы 5\\
\hline
6 & $2 \le n \le 10$ & $1 \le t \le F(n)$& 20 & Все тесты групп 1, 3, 5\\
\hline
7 & $2 \le n \le 1000$ & $1 \le t \le 10^4$& 37 & Все тесты групп 1--7\\
\hline
\end{tabular}

\Example

\begin{example}
\exmpfile{example.01}{example.01.a}%
\end{example}

\end{problem}

