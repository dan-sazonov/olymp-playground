\begin{problem}{Наибольший общий делитель}{стандартный ввод}{стандартный вывод}{4 секунды}{256 мегабайт}

Наибольшим общим делителем непустого набора натуральных чисел $A$ называется максимальное натуральное число $d$, такое что оно является одновременно делителем всех чисел множества $A$.

Задан массив натуральных чисел $[a_1, a_2, \ldots, a_n]$ и число $k$. Требуется выбрать в нем подмассив из $k$ подряд идущих элементов $[a_l, a_{l+1}, \ldots, a_{l+k-1}]$, чтобы их наибольший общий делитель был как можно больше, и вывести этот наибольший общий делитель.

\InputFile
Первая строка ввода содержит два целых числа $n$ и $k$ ($2 \le n \le 500\,000$, $2 \le k \le n$).

Вторая строка содержит $n$ натуральных чисел $a_1, a_2, \ldots, a_n$ ($1 \le a_i \le 10^{18}$).

\OutputFile
Выведите одно натуральное число "--- максимальное возможное значение наибольшего общего делителя элементов подмассива длины $k$ заданного массива.

\Scoring
Тесты в этой задаче разбиты на 7 групп. Баллы за группу начисляются только если
все тесты этой группы и всех необходимых групп пройдены.

\medskip

\begin{tabular}{|c|ccc|c|l|}
\hline
№ & \multicolumn{3}{c|}{Ограничения}& Баллы & Условие начисл. баллов\\
\hline
1 & $2 \le n \le 100$& $k=2$ &$1\le a_i\le 100$& 9 & Все тесты группы 1\\
\hline
2 & $2 \le n \le 100$& $k = 2$&$1\le a_i\le 10^9$& 9& Все тесты групп 1 и 2\\
\hline
3 & $2 \le n \le 100$& $k = 2$&$1\le a_i\le 10^{18}$& 9& Все тесты групп 1, 2 и 3\\
\hline
4 & $2 \le n \le 100$ & $2\le k \le n$&$1 \le a_i \le 100$& 10 & Все тесты групп 1 и 4\\
\hline
5 & $2 \le n \le 100$ & $2\le k \le n$&$1 \le a_i \le 10^{18}$& 10 & Все тесты групп 1, 4 и 5\\
\hline
6 & $2 \le n \le 500\,000$ & $2 \le k \le n$&$1 \le a_i \le 100$& 20 & Все тесты групп 1, 4 и 6\\
\hline
7 & $2 \le n \le 500\,000$ & $2 \le k \le n$&$1 \le a_i \le 10^{18}$& 33 & Все тесты групп 1--7\\
\hline
\end{tabular}

\Example

\begin{example}
\exmpfile{example.01}{example.01.a}%
\end{example}

\end{problem}

