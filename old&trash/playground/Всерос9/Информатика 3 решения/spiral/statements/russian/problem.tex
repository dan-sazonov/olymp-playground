\begin{problem}{Спираль}{стандартный ввод}{стандартный вывод}{1 секунда}{256 мегабайт}

Робот перемещается по клетчатой плоскости и рисует спираль. Исходно он находится в клетке~$(0, 0)$ и направлен в сторону увеличения первой координаты. 

Далее он действует по следующему алгоритму: совершает $d$ перемещений вперед, затем поворачивает налево и снова делает $d$ перемещений вперед. После этого он поворачивает налево и умножает значение $d$ на $k$. Затем робот повторяет описанный процесс. Робот останавливается, сделав суммарно ровно $n$ перемещений.

Требуется вывести картинку, на которой отмечены клетки, на которых побывал робот.

\InputFile
На вход подаются целые числа $n$, $d$ и $k$ ($1 \le n \le 1000$, $1 \le d \le 100$, $2 \le k \le 5$).

\OutputFile
Пусть минимальный прямоугольник из клеток, содержащий все посещенные роботом клетки, имеет высоту $h$ и ширину $w$. На первой строке выведите числа $h$ и $w$, разделенные пробелом. Следующие $h$ строк должны содержать по $w$ символов, выведите <<\texttt{*}>> для клетки, посещенной роботом и <<\texttt{.}>> для не посещенной.

\Scoring
В этой задаче 20 тестов, каждый из которых оценивается независимо в 5 баллов.

\Example

\begin{example}
\exmpfile{example.01}{example.01.a}%
\end{example}

\end{problem}

